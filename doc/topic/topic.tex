% Document style
\documentclass[a4paper,12pt]{article}
\usepackage[utf8]{inputenc}   % Czech
\usepackage[pdftex]{graphicx}  % images
\usepackage[czech]{babel}
\usepackage{enumerate}
\usepackage{hyperref}

% Begin
\begin{document}
\begin{titlepage}

% FI logo
\begin{center}
	\includegraphics[keepaspectratio]{fi-logo}
\end{center}

\vfill % vertical center
	\begin{center}
		\LARGE{Návrh na zadání diplomové práce}\\
		\vfill
		\bf{Organizace lidí v internetovém prostředí napříč různými komunikačními kanály}
	\end{center}
\vfill

% Author
\noindent
Autor: {\bf Jan Javorek} ({\tt 208013})\\
Datum: \today \\
Fakulta Informatiky\\
Masarykova Univerzita

\end{titlepage}

% TOC
%\tableofcontents
%\newpage

% Introduction
%\section{Úvod}
%Úvod.
%\newpage

% Contents
\section{Motivace}
Potřebujeme-li organizovat lidi, např. při pořádání nějaké události nebo domluvě na společném rozhodnutí, staneme před problémem jak je systematicky kontaktovat a jak efektivně sdílet dohodnuté informace. Můžeme sice využít e-mailů, instantních zpráv nebo sociálních sítí, ale brzy zjistíme, že taková komunikace je velmi roztříštěná -- lidé se totiž neradi přizpůsobují nebo registrují do nových služeb a tak skončíme v situaci, kdy polovinu známých organizujeme přes Facebook, několik jednotlivců přes e-maily a zbytek snad
přes Google Kalendář.

Pořadateli by v tomto případě jiste přišla vhod služba, která by u\-mož\-ňo\-va\-la komunikovat s lidmi uceleně přes několik kanálů. Služba centralizující organizaci, ale zachovávající svobodu jednotlivců rozhodnout se pro takový komunikační kanál, jenž vyhovuje jim. Organizátor by přes zmíněný {\it velín} mohl psát svým kontaktům zprávy a udržovat jejich historii, vytvářet jednoduché stránky se souhrnem dohodnutých informací, činit rozhodnutí na anketách, nebo zakládat {\it události}, u nichž by lidé mohli potvrdit či zamítnout účast.

Přitom běžný účastník domluvy by používal nástroje, na jaké je zvyklý a do organizační služby by se nemusel nijak registrovat nebo přihlašovat. Jestliže má účet na Facebooku, mohl by s organizátorem komunikovat pro\-střed\-nic\-tvím tohoto kanálu, pokud mu vyhovuje Google se svým kalendářem nebo prosté e-maily, mohl by je použít zrovna tak.

\section{Návrh zadání}
Internet je fragmentovaný a decentralizovaný i dnes, v době velkých sociálních sítí typu Facebook. Málokdy se lidé při společné domluvě sejdou na stejné platformě a proto se často uchylují k primitivním řešením jako je e-mail. Tím se však připravují o pohodlí, o možnost využít anket, událostí apod. Je možné vytvořit aplikaci, která by pomohla lidi efektivněji a pohodlněji organizovat, ale přitom jim nevnucovala žádnou konkrétní službu či nové uživatelské účty?

\begin{itemize}
    \item Prostudujte problematiku běžné organizace malých skupin lidí přes internet. Zjistěte, jaké nástroje nebo služby lidé již využívají pro komunikaci a vzájemnou organizaci (e-mail, instant messaging, sociální sítě). Proveďte průzkum ve svém okolí, např. sledováním chování zástupců různých druhů uživatelů nebo dotazníkem.
    \item Navrhněte řešení, jak propojit používané služby tak, aby mezi sebou lidé mohli komunikovat a sdílet organizační informace i bez toho, že by používali stejné komunikační platformy. Navrhněte jak řešit sdílení
    zpráv, dohodnutých rozhodnutí, anket, událostí s možností zadávat RSVP ({\it répondez s'il vous plaît}), apod. Podrobně rozpracujte každý z těchto případů a snažte se o nalezení takového východiska, které bude brát vždy největší ohledy na přirozenost a zřejmost používání (zvažte např. možnosti udržování vláken diskusí napříč všemi systémy apod.). Vysvětlete, proč jsou některé tyto problémy nerealizovatelné, narazíte-li na nějaké takové.
    \item Realizujte ukázkovou webovou aplikaci uvedených vlastností.
    \item Zhodnoťte perspektivy a další možné aplikace takového řešení. Zamyslete se nad možnostmi propojení dalších služeb nebo komunikačních kanálů.
\end{itemize}

% End
\end{document}
