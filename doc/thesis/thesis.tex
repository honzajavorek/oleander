\documentclass[12pt,oneside,final]{fithesis2}
\usepackage[czech]{babel}
\usepackage[utf8]{inputenc}
\usepackage[T1]{fontenc}
\usepackage{graphicx}
% \usepackage{url}
\usepackage[plainpages=false,pdfpagelabels,unicode]{hyperref}



\thesistitle{Organizace lidí v internetovém prostředí napříč různými komunikačními kanály}
\thesissubtitle{Diplomová práce}
\thesisstudent{Bc. Jan Javorek}
\thesiswoman{false}
\thesisfaculty{fi}
\thesisyear{jaro 2012}
\thesisadvisor{Mgr. Fedor Tiršel}
\thesislang{cs}



\begin{document}
\FrontMatter
\ThesisTitlePage



\begin{ThesisDeclaration}
\DeclarationText
\AdvisorName
\end{ThesisDeclaration}



%\begin{ThesisThanks}
%I would like to thank my supervisor...
%\end{ThesisThanks}



%\begin{ThesisAbstract}
%The aim of the bachelor work is to provide...
%\end{ThesisAbstract}



%\begin{ThesisKeyWords}
%keyword1, keyword2, etc.
%\end{ThesisKeyWords}



\tableofcontents



\MainMatter



% vědecká metoda:
%    - Herout: http://www.herout.net/blog/2012/03/struktura-diplomove-prace/#comment-476406168
%    - Wiki: https://cs.wikipedia.org/wiki/V%C4%9Bdeck%C3%A1_metoda



\chapter{Úvod}
Jak píše docent Adam Herout:
\begin{itemize}
    \item \url{http://www.herout.net/blog/2009/01/zase-postreh-ze-cteni-sp/}
    \item \url{http://www.herout.net/blog/2011/12/dve-poznamky-k-semestralnim-projektum-jakoz-i-diplomkam/}
\end{itemize}

Nemá se jednat o „úvod do problematiky“, ale o „úvod do knížečky“. Po jeho přečtení tedy má čtenář 1) mít představu o čem knížečka bude, 2) se těšit na to, že si ji přečte. Úvod ať se vejde na jednu stranu a nemá podkapitoly

Úvod v diplomce je úvod do vašeho textu, ne úvod do problematiky. Čtenář se má dozvědět, jaký je účel a cíl vaší práce, s čím má přistoupit k textu, co má a nemá očekávat.

Není to úvod do problematiky, kde by se čtenář dozvěděl, kde se vzaly technologie a techniky, které používáte, to přijde později, to je právě ta tzv. teoretická část.

Úvod se má vejít maximálně na jednu stranu, spíš na půl. Ještě jsem neviděl dobře napsaný úvod, který by mělo smysl strukturovat do podkapitol – pouze do odstavců.

Opakuji: Úvod je úvod k textu diplomky, ne úvod do problematiky.

Z úvodu má čtenář zjistit, co se v práci dozví a jakým způsobem, ale ještě se nic z toho nemá dozvědět. Úvod se tedy má vejít na 1/2 – 1 stranu, nemá být strukturován do podkapitol, pouze odstavců a má dát velice stručně tyto informace: Čím se práce zabývá a proč je to důležité, co všechno se člověk čtením textu dozví a jaká je struktura práce.

Terminologie, shrnutí použité teorie a tyto věci už patří do kapitoly po úvodu. Recenzent čte úvod hned na začátku a je jím tedy naladěn na práci – těší se, nebo je předem naštván; nechte si na úvodu záležet! Máločím své práci ublížíte tak účinně, jako úvodem, který sestává z nicneříkající vaty, nepodložených siláckých tvrzení, něčeho jakože vtipného a podobně.

\chapter{Motivace}
Shrnutí důvodů pro práci na takovéto téma. Uvedení do problematiky, popis současné situace. Upozornění na roztříštěnost komunikačních a organizačních nástrojů, stručná minulost a nastínění možné budoucnosti.

Co už v oblasti mého zadání existuje? Jak to dělají jiní?

Přečíst:
\begin{itemize}
    \item \url{http://www.amazon.com/Designing-Social-Web-ebook/dp/B0015DWIQ0/ref=tmm_kin_title_0?ie=UTF8&m=A2M9W3KSQUCQTK}
    \item \url{http://is.muni.cz/th/143212/fi_m/}
\end{itemize}

\chapter{Analýza způsobů organizace lidí přes internet}
Kapitola má za úkol analyzovat způsoby, jakými lidé přes internet běžně
organizují malé skupiny lidí (e-mail, instant messaging, sociální sítě,
VoIP, atd.). Výčet způsobů by měl být podložen průzkumem či výsledky dotazníku mezi lidmi.

\chapter{Identifikace elementárních prvků komunikace}
Kapitola se snaží rozložit jednotlivé způsoby komunikace na elementární
prvky jako např. zpráva, vlákno, rozhodnutí, aj., které mají potenciál být
integrovány (např. události na Facebooku a záznamy v Google Kalendáři).

\chapter{Specifikace požadavků a rámcový návrh řešení}
Cílem této části je shrnout výzkum předešlých dvou kapitol a rámcově navrhnout řešení aplikace. Výsledkem této kapitoly by měla být především specifikace zadání, jeho doplnění, konkretizace toho co přesně bude implementováno v aplikaci a proč právě to (cílení projektu, realizovatelnost, apod.).

Zadaný problém by šel řešit tak a nebo tak, já k němu přistoupím tímto způsobem, protože na zvolené platformě je to nejefektivnější. Rozhodl jsem se. Vymyslel jsem. Rozvrhl jsem. Vypočítal jsem. Odvodil jsem. Zjednodušil jsem. Vylepšil jsem. Navrhl jsem. Zjistil jsem. Vyzkoumal jsem.

\chapter{Použité technologie}
Zde by mělo být popsáno a rozebráno jaké nástroje a služby budou použity pro implementaci aplikace. U každého nástroje či služby by mělo být stručně vysvětleno k čemu slouží a z jakého důvodu došlo k výběru.

\chapter{Návrh systému}
Přesný popis návrhu systému. Struktura, uživatelské rozhraní, model, atd.

\chapter{Implementace}
Pro implementaci jsem zvolil ty a ty nástroje, celý systém rozvrhl do takových modulů. Naprogramoval jsem. Posbíral jsem data. Pustil jsem to. Výsledky jsou takové. Je to tak a tak rychlé.

Popis výsledné implementace. Část kapitoly by se měla zaměřit na výběr zajímavých problémů, na které se při implementaci narazilo, spolu s jejich řešeními.

\chapter{Vývoj a testování}
Zpráva o tom jak probíhal vývoj aplikace (systém správy verzí, aj.) a jak byla testována (na kterých prohlížečích, platformách, co bylo testováno, co testováno nebylo a proč, unit testy, apod.).

\chapter{Zkušenosti z provozu}
Pokud dojde na včasné spuštění služby, bylo by možno napsat kapitolu o zkušenostech z jejího provozu. Systém může být skvěle naprogramovaný, ale nakonec se např. přijde na to, že lidé stejně mají nějaké zábrany jej z určitých důvodů používat (viz Google Wave). V takovém případě lze v této části navrhnout nějaké změny, které by mohly kladnému přijetí pomoci.

Výsledek je takhle rychlý, má takovou úspěšnost a reakce uživatelů jsou takové a takové.

\chapter{Závěr}
Jak píše docent Adam Herout: \url{http://www.herout.net/blog/2009/01/zase-postreh-ze-cteni-sp/} nebo \url{http://www.herout.net/blog/2012/03/struktura-diplomove-prace/}.

Se závěrem se to má podobně jako s úvodem: má se vejít na 1/2 – 1 stranu, nemá být strukturovaný a má práci shrnout, uzavřít, zhodnotit. U školní práce je rozumné v závěru reagovat na zadání a shrnout bod po bodu, že byl splněn a kterou kapitolou práce (ovšem nenásilně – nikoli například odrážkami). Recenzent čte závěr na konci a závěr tedy z velké části utváří dojem zanechaný prací pro hodnocení – nechte si na něm záležet.

V závěru ať nepřicházejí žádné nové poznatky, neobjeví se tam nové číslo nebo nový graf.


% \bibliographystyle{plain}
% \bibliography{bib-db}



\end{document}
